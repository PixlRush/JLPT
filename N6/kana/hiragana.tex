\section[ひらがな]{ひらがな}\label{sec:N6;かな;ひらがな}

Hiragana is the main syllabary used for Japanese writing. Everything can be written entirely in hiragana, but the other scripts are just as important. Below, are the basic letters used for hiragana writing. Note that this table is read right to left, top to bottom.

\input{N6/kana/resources/hiragana-gojuuon}

This table is constructed to mirror how R\=omaji works. The kana そ, which lies at the intersection of S and O, is read `so.' The kana ん all on its own is that weird `n' talked about at the end of Section~\ref{sec:N6;発音;ローマ字}. There \textit{are} a few exceptions however. し is read as `shi,' ち is read as `chi,' and つ is read as `tsu.' This means that `si,' `ti,' and `tu' all can not be written using these basic kana.

There are a few more things to touch on past the basic kana. They can be marked in a way that changes their sounds as shown below:

\begin{center}
\begin{tabular}{|ccccc|c|}
\hline
P & B & D & Z & G &    \\\hline
ぱ & ば & だ & ざ & が & A \\
ぴ & び & ぢ & じ & ぎ & I \\
ぷ & ぶ & づ & ず & ぐ & U \\
ぺ & べ & で & ぜ & げ & E \\
ぽ & ぼ & ど & ぞ & ご & O \\
\hline
\end{tabular}
\end{center}


Much like with the original set of hiragana, じ, ぢ, and づ all do not fall nicely into the pattern. However ず is also included in this. じ, and ぢ are ponounced `ji'. づ, and ず are pronounced `zu.' Though do remember that this is the exceptional `zu' where it is pronounced like the `ds' in `goods' followed by the `u' vowel.

Lastly we have the rest of the sounds, those are formed by sticking a small Y-column kana at the end. They are responsible for a lot of the inserted `y's in Section~\ref{sec:N6;発音;ローマ字}. They are shown below:

\begin{center}
\begin{tabular}{|cccc|ccccccc|c|}
\hline
PY & BY & J\footnotemark   & GY & RY & MY & HY & NY & CH & SH & KY &    \\\hline
ぴゃ & びゃ & じゃ              & ぎゃ & りゃ & みゃ & ひゃ & にゃ & ちゃ & しゃ & きゃ & A \\
ぴゅ & びゅ & じゅ              & ぎゅ & りゅ & みゅ & ひゅ & にゅ & ちゅ & しゅ & きゅ & U \\
ぴょ & びょ & じょ              & ぎょ & りょ & みょ & ひょ & にょ & ちょ & しょ & きょ & O \\
\hline
\end{tabular}
\end{center}

\footnotetext{Note that you may also represent these sounds by swapping じ for ぢ. However this is the rarest kana.}

This now covers all the main hiragana. As it is the evergreen syllabary, the use of katakana will mainly be in contrast to the use of hiragana.
