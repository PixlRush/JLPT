\section[カタカナ]{カタカナ}\label{sec:N6;かな;カタカナ}

Katakana functions identically to hiragana in its structure. The charts, identical in structure to hiragana, are shown below. Note that the sound shifts in Section~\ref{sec:N6;かな;ひらがな} are also present.

\begin{center}
\begin{tabular}{|ccccc|c|cccccccccc|c|}
\hline
P & B & D & Z & G & N & W & R & Y & M & H & N & T & S & K & - &   \\\hline
パ & バ & ダ & ザ & ガ &   & ワ & ラ & ヤ & マ & ハ & ナ & タ & サ & カ & ア & A \\
ピ & ビ & ヂ & ジ & ギ &   &   & リ &   & ミ & ヒ & ニ & チ & シ & キ & イ & I \\
プ & ブ & ヅ & ズ & グ & ン &   & ル & ユ & ム & フ & ヌ & ツ & ス & ク & ウ & U \\
ペ & ベ & デ & ゼ & ゲ &   &   & レ &   & メ & ヘ & ネ & テ & セ & ケ & エ & E \\
ポ & ボ & ド & ゾ & ゴ &   & ヲ & ロ & ヨ & モ & ホ & ノ & ト & ソ & コ & オ & O \\
\hline
\end{tabular}
\end{center}


\begin{center}
\begin{tabular}{|cccc|ccccccc|c|}
\hline
PY & BY & J   & GY & RY & MY & HY & NY & CH & SH & KY &    \\\hline
ピャ & ビャ & ヂャ & ギャ & リャ & ミャ & ヒャ & ニャ & チャ & シャ & キャ & A \\
ピュ & ビュ & ヂュ & ギュ & リュ & ミュ & ヒュ & ニュ & チュ & シュ & キュ & U \\
ピョ & ビョ & ヂョ & ギョ & リョ & ミョ & ヒョ & ニョ & チョ & ショ & キョ & O \\
\hline
\end{tabular}
\end{center}


However, there are additional kana that are used most commonly in katakana. Katakana tends to be used more for transcribing words and sounds that are not from Japanese itself. Things like loanwords and sound effects. This leads to an entirely new structure of sticking small vowels next to some sound shifted letters to find a way to write sounds that we previously couldn't in Japanese. Take the hit game series 「ファイナル・ファナタジー」, typically written in English as `Final Fantasy.' We would pronounce that as `fainaru fantajii.' That 「ファ」 is read as `fa.'

The other thing to talk about that katakana tends to do differently from hiragana, is the 「ー」. This is called the vowel extension mark, or 「ちょうおんぷ」 \textit{(ch\=oonpu)}. It causes the vowel before it to be extended. This can also be used in hiragana as is typically seen in the word 「らーめん」, though that is a loanword from Chinese that has been adapted into Japanese.

The table for long and short vowels is shown below, with the entries being written in the script that they are most commonly found in.

\begin{center}
\begin{tabular}{|cc|c|c|}
\hline
\multicolumn{2}{|c|}{Always Long} & Sometimes Long &\\\hline
アー & ああ &      & \=A \\
イー & いい &      & \=I \\
ウー & うう &      & \=U \\
エー & ええ & えい & \=E \\
オー & おお & おう & \=O \\
\hline
\end{tabular}
\end{center}
