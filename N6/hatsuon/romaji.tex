\section[ローマ字]{ローマ\ruby{字}{じ}}\label{sec:N6;発音;ローマ字}

This section will introduce the concept of ローマ\ruby{字}{じ}, which when written in ローマ\ruby{字}{じ}~is rendered as `R\=omaji.' This is the method of writing Japanese words using Roman characters. Hence the name, Rome letters. We are starting with this, as Chapter~\ref{chp:N6;かな} will introduce the Japanese scripts in relation to Romaji\footnote{Specifically, we will be using Hepburn R\=omaji, the Kunrei system will be touched on briefly in Chapter~\ref{chp:N6;かな}}.

To begin with, let's talk about how Romaji works. There are five vowel sounds, and many consonant sounds. For the most part, they are pronounced exactly the same as you would see in English. The most important part when talking about pronounciation is that each sound is pronounced the same way when it shows up, and that there is no intonation - that is to say it is a `flat' language\footnote{This is not entirely true, it is a useful simplification for teaching, the truest series of answers will be revealed over time.}.

First off, let's talk about the five vowel sounds:

\begin{itemize}

	\item \textbf{a} --- pronounce it like the `ow' in `how', however only take the first part of that `ow'. As you say the word `how', note how you glide between two sounds for that `ow', only say that first half that sits lower in your mouth and before you start to round you lips.

	\item \textbf{i} --- pronounce it like the `ee' in `meet'

	\item \textbf{u} --- pronounce it like the `oo' in `shoot'

	\item \textbf{e} --- pronounce it like the `e' in `bet'

	\item \textbf{o} --- pronounce it like the `o' in `story', however take only the first part of that `o'. As you say the word `story' note how you glide between two sounds for that `o', only say that first half that sits more back in your mouth and before you start to round your lips.

\end{itemize}

Most of them are identical to English sounds you know, just make sure for the a and o to only get that first half of that English sound hint. However, the vowels here can exist in short and long variants. We differentiate them by using a line over the long vowels, the only exception will be a long-i sound, that is transcribed as `ii.' So they will be transcribed as: `\=a', `ii', `\=u', `\=e', `\=o.' A long vowel is pronounced identically to the short vowel, just takes double the time to say.

Now let us move onto the consonants, I will be going through them in groups. First off, the simple single consonants:

\begin{itemize}

	\item \textbf{k} --- pronounce this like the `k' in `skate'

	\item \textbf{g} --- pronounce this like the `g' in `again'

	\item \textbf{s} --- pronounce this like the `s' in `soup'

	\item \textbf{z} --- pronounce this like the `z' in `zoo'

	\item \textbf{t} --- pronounce this like the `t' in `stop'

	\item \textbf{d} --- pronounce this like the `d' in `today'

	\item \textbf{n} --- pronounce this like the `n' in `not'

	\item \textbf{h} --- pronounce this like the `h' in `hat'

	\item \textbf{b} --- pronounce this like the `b' in `about'

	\item \textbf{p} --- pronounce this like the `p' in `span'

	\item \textbf{m} --- pronounce this like the `m' in `much'

	\item \textbf{r} --- pronounce this like a Canadian would say the `d' in `trader,' this is the hardest sound to get right, it is a quick slap of that ridge in your mouth, a normal English `r' works fine for the most part but that is the most correct way to say it.

	\item \textbf{y} --- pronounce this like the `y' in `yacht'

\end{itemize}

Now for the weird consonant clusters:

\begin{itemize}

	\item \textbf{sh} --- pronounce it like the `sh' in `sheep'

	\item \textbf{ch} --- pronounce it like the `ch' in `cheap'

	\item \textbf{j} --- pronounce it like the `j' in `jeep'

	\item \textbf{ts} --- pronounce it like the `ts' in `cats'

	\item \textbf{z}u --- pronounce it like the `ds' in `cards,' note that this applies only for the `z' followed by a `u'

	\item \textbf{f} --- pronounce it like the `ph' in `phew!'

\end{itemize}

Lastly some consonants can have a `y' stuck on the end of them, here they are:

\begin{itemize}

	\item \textbf{ky} --- pronounce it like the `k' in `skew'

	\item \textbf{gy} --- pronounce it like the `g' in `argue'

	\item \textbf{ny} --- pronounce it like the `ny' in `canyon'

	\item \textbf{by} --- pronounce it like the `b' in `rebuke'

	\item \textbf{py} --- pronounce it like the `p' in `spew'

	\item \textbf{my} --- pronounce it like the `m' in `mute'

	\item \textbf{ry} --- just like with the solo `r,' this one is hard to do justice in words, try to go straight from the `r' to a `y' sound

\end{itemize}

Then you have the weird one allocated to the very very end. The solo `n'. It is hard to accurately describe how to pronounce it, as its pronounciation changes based on its context. It is most times just a generic nasal sound, think of something like `mmmmm' but open your mouth while saying it. To avoid confusion when this `n' is next to a `y', we may differentiate like so: `kon'yaku' vs `ny\=ugaku.'