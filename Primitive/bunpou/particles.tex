\section[助詞]{\ruby{助}{じょ}\ruby{詞}{し}}\label{sec:PR;文法;助詞}

Particles are helper words, the literal translation of \ruby{助}{じょ}\ruby{詞}{し}. They go at the end of a word or phrase to mark it for a whole bunch of grammatical functions. They are a big grammatical jump from English's word order, and will feel weird until you get used to them. This will talk about the primitive function of seven of the most common particles, show example sentences, and give you an idea of how to use them. To better show you word and particle boundaries, spaces will be added between them. Additionally it will all be written without the use of kanji. This is not how text is normally written, but this is useful to teach what is going on here.

\subsection*{は(わ)}\label{ssec:PR;文法;助詞;は}

Starting off with the first of two particles that will cause a headache, the は particle. First off, its pronounciation. It is the hiragana は pronounced as わ. This particle defines the topic of a sentence or string of sentences. The most natural English translation of 「〇〇 は」 would be `as for 〇〇.'

Typically you will find this particle at the start of a sentence or clause. Much like in the classic 「これ は ペン です」 sentence. Literally translated, it comes out to `As for this \textit{(thing)}, \textit{(it)} is \textit{(a)} pen.' Translating it for meaning not structure you would say `This is a pen.'

This は can exist on top of almost any other particle to modify and promote it to the status of topic. However when it comes to the particles が and を, it flat out replaces them. This will be further discussed below in the が particle's section to better compare and contrast them.

\subsection*{が}\label{ssec:PR;文法;助詞;が}

This particle is the other half of the は/が headache. This marks the subject of a sentence. The subject is the main thing being talked about in a sentence. For example, in the sentence 「それ が かわいい」. The subject is what comes before が, so in this case it would be `that.' The other word in the sentence is `cute,' which in this case would be describing the subject. So we could translate this as `That is cute.'

There are a few minute differences between these particles that will be touched on later, but these first two together are the hardest set of particles to explain exactly how they work. Especially because は can replace が in certain situations.

\subsection*{を(お)}\label{ssec:PR;文法;助詞;を}

This particle marks the direct object of a verb. That is, the thing that the verb is directly acting on. Think about the verb to eat, you have to eat something. Look at the sentence 「すし を たべる」. It translates to `???? eat sushi.' but is most commonly understood as `I eat sushi.' You could use が to specify what the subject is, so you could write 「たなかさん が すし を たべる」 to translate to `Tanaka eats sushi.'

\subsection*{に}\label{ssec:PR;文法;助詞;に}

This particle marks a destination to go to, it also can be translated as any of: to, in, at, or by. A simple example could be 「がっこう に いる」, literally `is at school' more accurately `I am at school.' Another example in the same vein is 「がっこう に いく」, `\textit{(I)} go to school.' One more example is 「おれ に かえせ」, `Give it back to me.' However, in this example we see に's additional function. It marks the indirect object of a verb.

To see an example of this in action, we can look at the sentence 「たなかさん に それ を くれ」. To literally translate, it would read `To Tanaka this give.' To accurately translate it would read `Give this to Tanaka.'

One final function, に also lets you set a time. Like to say `Tomorrow I will eat sushi' you would write 「あした に すし を たべる」. あした means tomorrow, but by marking it with に it sets a time for when things will happen in the sentence.

\subsection*{か}\label{ssec:PR;文法;助詞;か}

This is the question marking particle. Yes there is a question mark, this will also serve that function. It will go at the end of the sentence to mark it as a question. If we modify this sentence slightly, we can turn it into a question「あした に すし を たべる か?」. `Will you eat sushi tomorrow?' is its most common translation.

This can also be used to mark the end of a word or clause. Like in the sentence 「すし か ラーメン か どっち を たべる か。」. Literally translated it would be `Sushi? ramen? which should I eat?'\footnotemark Also look at how there is a 「。」 at the end of the sentence, you don't need to always pair a「か」 with a 「?」. Looking a little deeper into that example sentence, you see that it can be used in a sort of listing fashion. Like you could translate that sentence as `Sushi or ramen, which should I eat?' and it would make complete sense. 

\footnotetext{In this case, read the first two `?'s not as sentence-ending punctuation, but how you make your voice higher when asking someone a question.}

\subsection*{ね}\label{ssec:PR;文法;助詞;ね}

This is another end of sentence particle, this one is a lot more exclusively found at the end of sentences. This translates perfectly to English as `eh?' when used at the end of a sentence. `It's a little cold eh?' translates near directly to 「ちょっと さむい ね?」.

\subsection*{の}\label{ssec:PR;文法;助詞;の}

This one functions as a possessor, think of it as an `'s' for Japanese. `My pen' would be written as 「わたし の ペン」. For this particle, it is a little more strict for how you place it in a sentence. What possesses what should be on either end of the の.

You can also use it as a sort of clarifier. Where you can say 「くるま の トヨタ」, this isn't `The car's Toyota' but rather `The Toyota that car' or `The Toyota of the cars.' This is more commonly seen in some older and more famous Japanese names, like 「ふじわらのもこう」. This is explicitly saying, that is もこう from the ふじわら family. 
