\section[動詞]{\ruby{動}{どう}\ruby{詞}{し}}\label{sec:PR;文法;動詞}

Verbs are the first major type of words that can be conjugated. This will talk about them as types of words and we will delve into the conjugations of them in Section \ref{sec:PR;文法;活用}. 

\subsection*{\ruby{一}{いち}\ruby{段}{だん}\ruby{動}{どう}\ruby{詞}{し}}\label{ssec:PR;文法;動詞;一段動詞}

These verbs are called \ruby{一}{いち}\ruby{段}{だん}~verbs because they have one stem from which they form words. This is the payoff for back in Section \ref{sec:PR;仮名;五十音図} when we named the rows and columns \ruby{段}{だん}~and \ruby{行}{こう}. All 一段 verbs end in 〜iる or 〜eる. Examples of them are \ruby{見}{み}る and \ruby{食}{た}べる. See how they end in 〜iる and 〜eる respectively? This class of verb is divided further into \ruby{上一段動詞}{かみいちだんどうし}~and \ruby{下一段動詞}{しもいちだんどうし}~which correspond one to one to the 〜iる and 〜eる verb types.

\subsection*{\ruby{五}{ご}\ruby{段}{だん}\ruby{動}{どう}\ruby{詞}{し}}\label{ssec:PR;文法;動詞;五段動詞}

These verbs are called \ruby{五}{ご}\ruby{段}{だん}~verbs because they have five stems from which they form words. Those five forms are called by English speakers the あ\ruby{形}{けい}, い\ruby{形}{けい}, え\ruby{形}{けい}, お\ruby{形}{けい}, and て\ruby{形}{けい}. This is because the verb will end in one of these five kana. All of these verbs will end in 〜u, and in some cases an 〜iる or 〜eる. Not all verbs that end in 〜iる or 〜eる are \ruby{一}{いち}\ruby{段}{だん}~, but all \ruby{一}{いち}\ruby{段}{だん}~ verbs end in 〜iる or 〜eる. Some examples of them are \ruby{飲}{の}む and \ruby{買}{か}う.

\subsection*{\ruby{変格}{へんかく}\ruby{動}{どう}\ruby{詞}{し}}\label{ssec:PR;文法;動詞;変格動詞}

There are two and only two verbs that don't fall neatly into one of the above two classes. They are the exceptions. \ruby{来}{く}る, `to come,' and する, `to do.' They will be talked about in more depth in the conjugations section, but it is good to bring them up as clear exceptions to the rules early.