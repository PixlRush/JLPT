\section[読む方法]{\ruby{読}{よ}む\ruby{方法}{ほうほう}}\label{sec:PR漢字読む方法}

Kanji have two major classes of readings: \ruby{音}{おん}\ruby{読}{よ}み and \ruby{訓}{くん}\ruby{読}{よ}み. Literally translated, those are the `sound reading' and `local/native reading' of a kanji. Typically you will write the kun'yomi in hiragana and the on'yomi in katakana.

Take for example the kanji 「三」. It has the on'yomi of さん and the kun'yomi of み、み(つ)、みっ(つ). The last two kun'yomi have a suffix attached to them. This is important as on it's own you will read 「三」 as さん but read 「三つ」 as みつ. Typically kanji will have one main on'yomi and kun'yomi however in some cases they will have many. In this case, I count it as having one kun'yomi.

There are however, other ways to read kanji. Most of these alternate readings require additional knowledge, but there is one that is useful to bring up now. This is a relic of older Japanese when these readings were correct, but as time went on and sounds shifted we got the modern readings for certain kanji and kanji combinations. Take for example the word you are likely familiar with: 「こんにちは」\footnotemark\addtocounter{footnote}{-1}. This is how you say hello to someone in Japanese, and it has the same etymology as the english phrase `good day.' However if you write it in its kanji form of 「今日は」\footnotemark\addtocounter{footnote}{-1}. You would read it instead as きょうは\footnotemark.
\footnotetext{Just like the case in Section \ref{sec:PR仮名五十音図}, the kana 「は」 is read as `wa' here, its a grammatical thing covered on page \pageref{subsec:PR文法助詞は}}

These kinds of archaic readings are ones that you just have to learn. As you go deeper and deeper you may eventually learn how and why this reading existed. But, for most of these you will just have to learn them as they come up.