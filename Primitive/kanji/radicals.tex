\section[部首]{\ruby{部首}{ぶしゅ}}\label{sec:PR;漢字;部首}

A very important step in understanding kanji is understanding radicals. These radicals are the pieces that kanji are made from. Different radicals will carry different pieces of information for their kanji. But, each kanji has one main radical, called the \ruby{部首}{ぶしゅ}. Here we will be using the Japanese term to refer to that main radical.

There are seven different types of radicals. They are split into two main types and will be explained below.

\subsection*{\ruby{左右}{さゆう}\ruby{上下}{じょうげ}は\ruby{偏旁}{へんぼう}\ruby{冠脚}{かんきゃく}}\label{ssec:PR;漢字;部首;左右上下は偏旁冠脚}

\hspace*{24pt}\textit{``Left and right, top and bottom, are henbou, and kankyaku.''}

These four radical types tend to come paired with one another. A left radical paired with a right, and a top radical paired with a bottom. We will go through these in the same order as above.

\subsection*{偏(へん)}\label{ssec:PR;漢字;部首;へん}

This is the left side radical, to show an example of it we may look at the kanji 偏 itself. In addition we will also look at the kanji 語. The highlighted and bolded sections are that へん radical. For the left one, we specifically call it にんべん - the person hen radical. The right one is called ごんべん.

\begin{figure}[H]\label{fig:PR;漢字;部首;へん}
	\centering
	\includesvg{Primitive/kanji/resources/svgs/radicals/偏にんべん}
	\hspace{0.5in}
	\includesvg{Primitive/kanji/resources/svgs/radicals/語ごんべん}
\end{figure}

\subsection*{旁(つくり)}\label{ssec:PR;漢字;部首;つくり}

This is the right side radical. Same idea as the left side radical above. Here are two examples of them, 数 and 対. The highlighted components are called - from left to right - the ぼくづくり and すんづくり.

\begin{figure}[H]\label{fig:PR;漢字;部首;つくり}
	\centering
	\includesvg{Primitive/kanji/resources/svgs/radicals/数ぼくづくり}
	\hspace{0.5in}
	\includesvg{Primitive/kanji/resources/svgs/radicals/対すんづくり}
\end{figure}

\subsection*{冠(かんむり)}\label{ssec:PR;漢字;部首;かんむり}

This is the top half radical. To take a look at two examples, we can use the kanji 字 and 電. The highlighted components are called - from left to right - the うかんむり and the あめかんむり. 

\begin{figure}[H]\label{fig:PR;漢字;部首;かんむり}
	\centering
	\includesvg{Primitive/kanji/resources/svgs/radicals/字うかんむり}
	\hspace{0.5in}
	\includesvg{Primitive/kanji/resources/svgs/radicals/電あめかんむり}
\end{figure}

\subsection*{脚(あし)}\label{ssec:PR;漢字;部首;あし}

Lastly, there is the bottom half radical. To take a look at two examples, we can use the kanji 恋 and 無. The highlighted components are called - from left to right - the こころ and the れんが.

\begin{figure}[H]\label{fig:PR;漢字;部首;あし}
	\centering
	\includesvg{Primitive/kanji/resources/svgs/radicals/恋こころ}
	\hspace{0.5in}
	\includesvg{Primitive/kanji/resources/svgs/radicals/無れんが}
\end{figure}

\subsection*{\ruby{囲}{かこ}まれるの\ruby{部首}{ぶしゅ}}\label{ssec:PR;漢字;部首;囲まれるの部首}

Instead of simple left and right compositions, we can have enclosing radicals. These ones either give a structure inside which other components will be placed, are placed over top of the kanji and hang down from the left side, or support the kanji from the bottom and the left sides.

\subsection*{構(かまえ)}\label{ssec:PR;漢字;部首;かまえ}

Going through the first type of enclosure radical, it literally translates to structure. These kinds of radicals will tend to either fully enclose another component as in 囲 or will enclose on three sides like in 間. Those two examples, which are shown below, have their かまえ components highlighted. They are called - from left to right - the くちがまえ and the もんがまえ.

\begin{figure}[H]\label{fig:PR;漢字;部首;かまえ}
	\centering
	\includesvg{Primitive/kanji/resources/svgs/radicals/囲くちがまえ}
	\hspace{0.5in}
	\includesvg{Primitive/kanji/resources/svgs/radicals/間もんがまえ}
\end{figure}

\subsection*{垂(たれ)}\label{ssec:PR;漢字;部首;たれ}

This kind of radical is the overhanging type. It will enclose its components from above and to the left. It is a pretty rare one, having only five radicals in its classification. To show two examples of it in action, we will look at 広 and 病. The highlighted components are called - from left to right - the まだれ and the やまいだれ.

\begin{figure}[H]\label{fig:PR;漢字;部首;たれ}
	\centering
	\includesvg{Primitive/kanji/resources/svgs/radicals/広まだれ}
	\hspace{0.5in}
	\includesvg{Primitive/kanji/resources/svgs/radicals/病やまいだれ}
\end{figure}

\subsection*{繞(にょう)}\label{ssec:PR;漢字;部首;にょう}

This is the last major class of radical, the supporting type. It will enclose its components from below and to the left. In some cases it will be treated as if it was a へん radical with a single stroke going under the つくり part, but those are mostly uncommon. Examples of these are found in 道 and 超. The highlighted components are called - from left to right - the しんにょう and the えんにょう.

\begin{figure}[H]\label{fig:PR;漢字;部首;にょう}
	\centering
	\includesvg{Primitive/kanji/resources/svgs/radicals/道しんにょう}
	\hspace{0.5in}
	\includesvg{Primitive/kanji/resources/svgs/radicals/超えんにょう}
\end{figure}