\section[伯]{\ruby{伯}{はく}}\label{sec:PR;仮名;伯}

Now that you have a vague understanding of how the kana form sounds, let's touch more precisely on how those sounds form words.

Each kana, sokuon, or youon-ed kana, will be pronounced for one mora. Japanese would refer to it as a `haku,' which would mean a musical beat. 「ひらがな」 has four kana so it will take four mora to pronounce, 「まって」 has three kana -- yes that sokuon counts as a kana for this counting -- so it will take three mora to pronounce, 「しょうしんしゃ」 has three kana and two youon pairs so it would take five mora to pronounce. By this rigid structure, Japanese has meaningful distinction between long and short vowels, as well as meaningful distinction between stopping for that sokuon or not. 「かれ」 is just some person, but 「かれー」is a healthy meal.

You know how the chouonpu works, however that's just one way to write the long vowels. To extend the length of a vowel without using a chouonpu, you just need to write the vowel kana you want to extend. 「ああ」, 「いい」, 「うう」, 「えい」, 「おう」 are how you extend all of the vowels. Now, `ei' and `ou' are the outliers here. You use a different sound to extend them, however they are pronounced identically to a long `e' or a long `o'. Reading them as is and gliding the vowel to the other one, or leaving it the same vowel are both valid readings.
