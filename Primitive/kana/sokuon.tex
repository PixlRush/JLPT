\section[促音]{\ruby{促}{そく}\ruby{音}{おん}}\label{sec:PR;仮名;促音}

While we are on the topic of small kana, let's bring up the final small kana that you will encounter. The sokuon, or as it is sometimes referred to, the small tsu.

This kana is pronounced in a very special way, it is a sort of pause between the sounds that you are saying. To use an example, 「ちょっとまって」 is pronounced as ``chotto matte.'' Note those duplicated `t's. You're saying that `t' twice in a sense, but it's much more silent. You are effectively saying a single syllable of \textit{nothing} before continuing with your word with a little bit more force. The precise mechanics and timing will be touched on in section \ref{sec:PR;仮名;伯}, but for now just get used to it as a small delay or a duplication of the previous sound.