\section[長音符]{\ruby{長}{ちょう}\ruby{音}{おん}\ruby{符}{ぷ}}\label{sec:PR;仮名;長音符}

You now know everything about the typical uses of kana, small kana, and how to construct sounds that you need. However there is one kana-related tool that serves to help the kana work. The chouonpu, literally ``long sound mark.'' It extends the duration of the vowel before it by one syllable. It's why 「ファイナルファンタジー」 was romanized as `fantajii' not just `fantaji.' This is most commonly seen in katakana loan words, like 「コンピューター」\textit{(`konpyuutaa')} or 「エスカレーター」\textit{(`esukareetaa')}. The precise mechanics will be touched on in section \ref{sec:PR;仮名;伯}, but it is important to bring it up now.
