\section[濁点と半濁点]{\ruby{濁点}{だくてん}と\ruby{半濁点}{はんだくてん}}\label{sec:PR;仮名;濁点と半濁点}

There is more to the story of kana than just the gojyuuonzu. Firstly, how do you even \textit{write} the `jyuu' and `zu' that are in that word? Those consonants don't appear anywhere in the gojyuuonzu. Both of those questions will be answered in the following sections, starting with ``how do you even write `zu?'''

Dakuten modify the sound of the kana they are placed on, literally meaning ``muddying mark.'' The dakuten is written with a 「゛」 and makes the sound voiced. That is if you were to say 「かかかか」 and 「がががが」 you would feel your vocal cords vibrating a lot more for the second one. Four kana take dakuten to become voiced, however one kana is special and can also take a handakuten. The handakuten -- literally ``half muddied mark'' -- only dilutes the sound somewhat. It looks like this: 「゜」. Below is the table of both hiragana and katakana with the dakuten and handakuten added on.

\begin{center}
\begin{tabular}{|ccccc|c|cccccccccc|c|}
\hline
P & B & D & Z & G & N & W & R & Y & M & H & N & T & S & K & - &   \\\hline
ぱ & ば & だ & ざ & が &   & わ & ら & や & ま & は & な & た & さ & か & あ & A \\
ぴ & び & ぢ & じ & ぎ &   &   & り &   & み & ひ & に & ち & し & き & い & I \\
ぷ & ぶ & づ & ず & ぐ & ん &   & る & ゆ & む & ふ & ぬ & つ & す & く & う & U \\
ぺ & べ & で & ぜ & げ &   &   & れ &   & め & へ & ね & て & せ & け & え & E \\
ぽ & ぼ & ど & ぞ & ご &   & を & ろ & よ & も & ほ & の & と & そ & こ & お & O \\
\hline
\end{tabular}
\end{center}


\begin{center}
\begin{tabular}{|ccccc|c|cccccccccc|c|}
\hline
P & B & D & Z & G & N & W & R & Y & M & H & N & T & S & K & - &   \\\hline
パ & バ & ダ & ザ & ガ &   & ワ & ラ & ヤ & マ & ハ & ナ & タ & サ & カ & ア & A \\
ピ & ビ & ヂ & ジ & ギ &   &   & リ &   & ミ & ヒ & ニ & チ & シ & キ & イ & I \\
プ & ブ & ヅ & ズ & グ & ン &   & ル & ユ & ム & フ & ヌ & ツ & ス & ク & ウ & U \\
ペ & ベ & デ & ゼ & ゲ &   &   & レ &   & メ & ヘ & ネ & テ & セ & ケ & エ & E \\
ポ & ボ & ド & ゾ & ゴ &   & ヲ & ロ & ヨ & モ & ホ & ノ & ト & ソ & コ & オ & O \\
\hline
\end{tabular}
\end{center}


Just like with the original goyjuuon, there are some exceptions to how to pronounce these kana. These are mostly sound mergers, where two sounds became pronounced the same but still need to be written differently. They are: ZI and DI are both read as ``ji'', and ZU and DU are both read as ``dzu.'' For the ZU, DU merger, you typically see it written as `zu' but I need to stress that it is the `tsu' that is getting voiced not the `su.'
