\section[五十音図]{\ruby{五十音図}{ごじゅうおんず}}\label{sec:PR仮名五十音図}

Much like how English has an alphabetical ordering of their letters, Japanese has it too. We call this ordering system the gojyuuon, ``fifty sounds'' in English. For it to be easier to read and use, we arrange it into a gojyuuonzu -- ``fifty sound map.'' This map is arranged into rows and columns. Rows are called `dan' (\ruby{段}{だん}) and columns are called `kou' (\ruby{行}{こう}). Both of those names will come back later in this book, so it is good to bring them up now rather than later.

With all this talk about the gojyuuonzu, let's show you one. Below is the hiragana gojyuuonzu, it should be read top to bottom and right to left. A kana is read by combining its column and row together and then pronouncing them. You would read 「いろはにほへと」 as ``irohanihoheto.'' 「は」 is at the intersection of H and A so it would be read as `ha.' This is true for \textit{almost} all of the kana, the exceptions are as follows: SI is read as `shi', TI is read as `chi', TU is read as `tsu', HU is read as a rather breathy `fu'. Then you have the single N, pronouncing that depends on context but it mostly sounds like a normal `n' if you pull your tongue away from that ridge in your mouth and let it sit on the bottom of your mouth. 

\begin{center}
\begin{tabular}{|c|cccccccccc|c|}
\hline
N & W & R & Y & M & H & N & T & S & K & - &   \\\hline
  & わ & ら & や & ま & は & な & た & さ & か & あ & A \\
  &   & り &   & み & ひ & に & ち & し & き & い & I \\
ん &   & る & ゆ & む & ふ & ぬ & つ & す & く & う & U \\
  &   & れ &   & め & へ & ね & て & せ & け & え & E \\
  & を & ろ & よ & も & ほ & の & と & そ & こ & お & O \\
\hline
\end{tabular}
\end{center}


Now let's talk about what hiragana is actually used for. Anything and everything you could want to write in Japanese can be written in hiragana. 「すしや は ここ に なります か?」 read as ``sushiya wa\footnotemark koko ni narimasu ka?'' Katakana on the other hand, has a lot more specific use cases. It is the script that you will be using to write loanwords. Think about all those English words that Japan just took and imported directly into Japanese. `One-oh-eight' is nothing more than 「ワンオウエイト」. It is also used to write down the sounds things make, the 「ワクワク」 of getting excited, the 「ワンワン」 that the dogs say, and so on. Below is the gojyuuonzu for katakana, reference it to read the katakana above.
\footnotetext{Yes that kana 「は」 is read as `wa' here, its a grammatical thing covered on page \pageref{subsec:PR文法助詞は}}

\begin{center}
\begin{tabular}{|c|cccccccccc|c|}
\hline
N & W & R & Y & M & H & N & T & S & K & - &   \\\hline
  & ワ & ラ & ヤ & マ & ハ & ナ & タ & サ & カ & ア & A \\
  &   & リ &   & ミ & ヒ & ニ & チ & シ & キ & イ & I \\
ン &   & ル & ユ & ム & フ & ヌ & ツ & ス & ク & ウ & U \\
  &   & レ &   & メ & ヘ & ネ & テ & セ & ケ & エ & E \\
  & ヲ & ロ & ヨ & モ & ホ & ノ & ト & ソ & コ & オ & O \\
\hline
\end{tabular}
\end{center}

