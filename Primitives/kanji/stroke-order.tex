\section[書き順]{\ruby{書}{か}き\ruby{順}{じゅん}}\label{sec:PR;漢字;書き順}

In the previous section you learned about radicals, the building blocks of kanji. This section will cover the stroke order of kanji, the most correct way to write them. These rules should serve as a guideline for how to write kanji, and will become more important as your handwriting gets more and more cursive. They can have exceptions that you should be aware of, but exceptions are rare. Each rule will have example kanji to show the rule in action.

\subsection*{\ruby{上}{うえ}から\ruby{下}{した}、\ruby{左}{ひだり}から\ruby{右}{みぎ}}\label{ssec:PR;漢字;書き順;上から下左から右}

\hspace*{24pt}\textit{``Top to bottom, left to right.''}

This is the most \textit{fundamental} rule when writing kanji. Horizontal strokes will be written starting on the left and going towards the right. Vertical strokes will be written starting from the top and going towards the bottom. Examples of these principles are shown below:

\begin{figure}[H]\label{fig:PR;漢字;書き順;上から下左から右}
	\centering
	\includesvg{Primitives/kanji/resources/svgs/stroke-order/三}
	\hspace{0.5in}
	\includesvg{Primitives/kanji/resources/svgs/stroke-order/川}
\end{figure}

\subsection*{\ruby{横}{よこ}の\ruby{後}{あと}に\ruby{縦}{たて}}\label{ssec:PR;漢字;書き順;横の後に縦}

\hspace*{24pt}\textit{``After the horizontal strokes, come the vertical strokes.''}

Unlike the above examples, most kanji have horizontal and vertical strokes. In this case, you should write the horizontal strokes before the vertical ones. An example of this principle are shown below:

\begin{figure}[H]\label{fig:PR;漢字;書き順;横の後に縦}
	\centering
	\includesvg{Primitives/kanji/resources/svgs/stroke-order/十}
\end{figure}


\subsection*{\ruby{刺}{さ}す\ruby{画}{かく}は\ruby{最後}{さいご}}\label{ssec:PR;漢字;書き順;刺す画は最後}

\hspace*{24pt}\textit{``Piercing strokes come last.''}

A piercing stroke is a stroke that goes through a bunch of others and spans a significant portion of a kanji. They should be written after the strokes they are piercing through. Examples of this principle are shown below:

\begin{figure}[H]\label{fig:PR;漢字;書き順;刺す画は最後}
	\centering
	\includesvg{Primitives/kanji/resources/svgs/stroke-order/中}
	\hspace{0.5in}
	\includesvg{Primitives/kanji/resources/svgs/stroke-order/母}
\end{figure}


\subsection*{\ruby{対角}{たいかく}\ruby{線}{せん}は\ruby{右上}{みぎうえ}が\ruby{左上}{ひだりうえ}より\ruby{前}{まえ}}\label{ssec:PR;漢字;書き順;対角線は右上が左上より前}

\hspace*{24pt}\textit{``For diagonal lines, the upper right goes before the upper left.''}

In the case where a kanji has multiple diagonal lines, draw the one that goes down and to the left before you draw the one that goes down and to the right. Examples of this rule are shown below:

\begin{figure}[H]\label{fig:PR;漢字;書き順;対角線は右上が左上より前}
	\centering
	\includesvg{Primitives/kanji/resources/svgs/stroke-order/人}
	\hspace{0.5in}
	\includesvg{Primitives/kanji/resources/svgs/stroke-order/文}
\end{figure}


\subsection*{\ruby{中心}{ちゅうしん}の\ruby{縦線}{じゅうせん}は\ruby{右左}{みぎひだり}の\ruby{翼}{つばさ}より\ruby{前}{まえ}に\ruby{書}{か}く}\label{ssec:PR;漢字;書き順;中心の縦線は右左の翼より前に書く}

\hspace*{24pt}\textit{``The central vertical line is drawn before any of the wings on either side.''}

This rule is very similar to the rule about piercing strokes. However these central strokes tend to not make contact with the elements on either side. Below is a simple example to show this principle at work, alongside a complex example that will combine all the other rules to form its stroke order:

\begin{figure}[H]\label{fig:PR;漢字;書き順;中心の縦線は右左の翼より前に書く}
	\centering
	\includesvg{Primitives/kanji/resources/svgs/stroke-order/水}
	\hspace{0.5in}
	\includesvg{Primitives/kanji/resources/svgs/stroke-order/金}
\end{figure}

For the 「金」 kanji, stepping through its stroke order one rule at a time has us come to the correct construction. Firstly, top to bottom, left to right. So we will start writing stroke 1 from the top-most point. Using the rule about diagonal lines, we draw stroke 1 down and to the left, snd stroke 2 down and to the right. Continuing down, we draw stroke 3. Stroke 4 comes before stroke 5 because stroke 5 is a piercing stroke. Strokes 6 and 7 are examples of wings. Lastly stroke 8 lays at the very bottom of the kanji, is not pierced, and is drawn last.


\subsection*{\ruby{左}{ひだり}の\ruby{縦線}{じゅうせん}は\ruby{右}{みぎ}か\ruby{下}{した}\ruby{線}{せん}\ruby{前}{まえ}}\label{ssec:PR;漢字;書き順;左の縦線は右か下線の前}

\hspace*{24pt}\textit{``The left vertical line comes before the right and bottom lines.''}

This is an important rule that has a powerful longer-term effect. People may also call it the box drawing rule, as this is the rule that causes boxes to be drawn as they are shown below:

\begin{figure}[H]\label{fig:PR;漢字;書き順;左の縦線は右か下線の前;口}
	\centering
	\includesvg{Primitives/kanji/resources/svgs/stroke-order/口}
	\hspace{0.5in}
	\includesvg{Primitives/kanji/resources/svgs/stroke-order/口強調}
\end{figure}

Note the important stroke in the second image, stroke number 2. This kind of pattern applies to almost every box shaped component of a kanji, as shown in more depth below:

\begin{figure}[H]\label{fig:PR;漢字;書き順;左の縦線は右か下線の前}
	\centering
	\includesvg{Primitives/kanji/resources/svgs/stroke-order/門}
	\hspace{0.5in}
	\includesvg{Primitives/kanji/resources/svgs/stroke-order/長}
\end{figure}

\subsection*{\ruby{囲}{かこ}む\ruby{線}{せん}は\ruby{囲}{かこ}まれる\ruby{線}{せん}より\ruby{前}{まえ}}\label{ssec:PR;漢字;書き順;囲む線は囲まれる線より前}

\hspace*{24pt}\textit{``The surrounding strokes come before the surrounded strokes.''}

This rule is a lot more understandable as: `The enclosing radical comes before the enclosed radical.' When a kanji contains an enclosing type radical - a かまえ or たれ type specifically - that enclosing radical must be drawn first. For an example, take a look at the following kanji:

\begin{figure}[H]\label{fig:PR;漢字;書き順;囲む線は囲まれる線より前}
	\centering
	\includesvg{Primitives/kanji/resources/svgs/stroke-order/間}
	\hspace{0.5in}
	\includesvg{Primitives/kanji/resources/svgs/stroke-order/国}
\end{figure}

However take a closer look at the stroke order for 「国」. The bottom line of the surrounding box is drawn after the entirety of the contents. This is the exception to the rule of draw the enclosing radical before the insides.

\subsection*{にょうがある\ruby{最後}{さいご}}\label{ssec:PR;漢字;書き順;にょうがある最後}

\hspace*{24pt}\textit{``If there is a にょう radical, write it last.''}

In the previous rule, note how I talked about two of the three types of surrounding radicals. The かまえ and たれ radicals specifically. This rule covers the last type of surrounding radical, the にょう radicals. Write those radicals last in stroke order. Examples below:

\begin{figure}[H]\label{fig:PR;漢字;書き順;にょうがある最後}
	\centering
	\includesvg{Primitives/kanji/resources/svgs/stroke-order/近}
	\hspace{0.5in}
	\includesvg{Primitives/kanji/resources/svgs/stroke-order/建}
\end{figure}

\subsection*{\ruby{点}{てん}は\ruby{最後}{さいご}}\label{ssec:PR;漢字;書き順;点は最後}

\hspace*{24pt}\textit{``If there is a dot, write it last.''}

This is the final rule. If the kanji contains a dot somewhere, that is a single small stroke that looks like this: 「ヽ」, write it last. Examples are shown below:

\begin{figure}[H]\label{fig:PR;漢字;書き順;点は最後}
	\centering
	\includesvg{Primitives/kanji/resources/svgs/stroke-order/玉}
	\hspace{0.5in}
	\includesvg{Primitives/kanji/resources/svgs/stroke-order/犬}
\end{figure}
